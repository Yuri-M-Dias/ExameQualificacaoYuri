%%%%%%%%%%%%%%%%%%%%%%%%%%%%%%%%%%%%%%%%%%%%%%%%%%%%%%%%%%%%%%%%%%%%%%%%%%%%%%%

\chapter{INTRODUÇÃO}

%Os capítulos restantes desta dissertação estão organizados da seguinte maneira:
%\begin{itemize}
%\item{Capítulo 2}: A Seção~\ref{seccaos} aborda, por meio de sistemas-paradigma de comportamento caótico como o Mapa Logístico e as Equações de Lorenz, as principais características apresentadas por sistemas dinâmicos não-lineares caóticos. A Seção~\ref{secturb} é destinada, em grande parte, à descrição fenomenológica da turbulência. São ainda apresentadas alguns temas de grande relevância no estudo da turbulência, como o fenômeno da intermitência e as estruturas coerentes. Finalmente, a Seção~\ref{seccaosurb} expõe uma perspectiva adicional para a compreensão do fenômeno turbulento com base na teoria de sistemas dinâmicos, mais especificamente por meio de atratores caóticos~\cite{ruelltak/71}.
%\item{Capítulo 3}: Neste capítulo é feita uma descrição detalhada dos métodos e dos algoritmos disponíveis para a caracterização de caos determinístico presente em séries temporais experimentais. O tema é bastante extenso e especializado; como conseqüência, tal descrição será focada nos procedimentos mais utilizados para a reconstrução da dinâmica, para o cálculo de dimensão de atratores e do espectro de expoentes de Lyapunov. São abordados alguns problemas e limitações, do ponto de vista numérico, associados aos algoritmos utilizados, como também as dificuldades encontradas no tratamento de sinais experimentais. 
%\item{Capítulo 4}: Neste capítulo é realizada uma breve descrição dos dados coletados pela campanha WETAMC do projeto LBA, bem como do sítio experimental.
%\item{Capítulo 5}: Neste capítulo são aplicadas diversas ferramentas (descritas no Capítulo~\ref{caputiltecnicas}) nas séries temporais em estudo, em busca de atratores caóticos de baixa dimensão na camada limite atmosférica.
%\item{Capítulo 6}: Com base nas análises realizadas no Capítulo~\ref{capanaliseseresults}, neste capítulo serão apresentadas as conclusões obtidas, como também algumas sugestões para trabalhos futuros.
%\end{itemize}

