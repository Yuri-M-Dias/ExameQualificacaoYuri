%%%%%%%%%%%%%%%%%%%%%%%%%%%%%%%%%%%%%%%%%%%%%%%%%%%%%%%%%%%%%%%%%%%%%%%%%%%%%%%

\chapter{CONCLUSÕES}


Neste trabalho foi analisada a possível natureza caótica da turbulência atmosférica. Trata-se de uma questão ainda em aberto para a qual existem resultados discrepantes. As análises aqui realizadas, baseadas em dados de temperatura de alta resolução, obtidos pela campanha WETAMC do projeto LBA, sugerem a existência de um comportamento caótico de baixa dimensão na camada limite atmosférica. O atrator caótico correspondente possui uma dimensão de correlação de $D_{2}=3.50\pm0.05$. A presença de dinâmica caótica nos dados analisados é confirmada com a estimativa de um expoente de Lyapunov pequeno mas positivo, com valor $\lambda_{1}=0.050\pm0.002$. No entanto, esta dinâmica caótica de baixa dimensão está associada à presença das estruturas coerentes na camada limite atmosférica e não à turbulência atmosférica, como anteriormente afirmado por vários autores~\cite{xin/01,jaramillo/93,gallego/01}. Esta afirmação é evidenciada pelo processo de filtragem por wavelets utilizado nos dados experimentais estudados, que permite separar a contribuição da estruturas coerentes do sinal turbulento de fundo.

Este resultado corrobora a conjectura de \citeonline{loratrat/91}, que afirma que as ligações existentes entre caos e clima, apresentadas na literatura, decorrem não de erros de análise, mas da existência de subsistemas caóticos de baixa dimensão fracamente acoplados a um sistema maior, mais complexo e não caótico. No contexto deste trabalho, os subsistemas de baixa dimensão seriam as estruturas coerentes, comumente encontradas na camada limite, e o sistema maior, a atmosfera complexa e turbulenta. Os resultados obtidos neste trabalho, tanto para a dimensão de correlação $D_{2}$ quanto para o expoente de Lyapunov $\lambda_{1}$, são consistentes com valores publicados na literatura, onde evidências de atratores de baixa dimensão na atmosfera foram detectados~\cite{xin/01,jaramillo/93,gallego/01}. Apesar da presença de estruturas coerentes não ser explicitamente mencionada nestes trabalhos, uma leitura cuidadosa dos mesmos fornece indícios de que as séries temporais estudadas por esses autores possuem estruturas do tipo rampa, como as analisadas neste trabalho.
 
No plano computacional, esta dissertação ofereceu a oportunidade de estudar e implementar diversas ferramentas para o estudo de comportamento caótico em séries temporais. Apesar de populares, a maior parte destes algoritmos apresenta sérias dificuldades quando confrontados com o desafio de analisar séries ``reais'' (isto é, oriundas de sistemas dinâmicos naturais não conhecidos a priori), fracamente estacionárias, com muitos pontos (100 mil ou mais) e ainda por cima, contaminadas com ruído. Neste sentido, um resultado importante desta dissertação foi o desenvolvimento de uma ``cultura'' local, composta por adaptações, truques, etc., que permitiu ir além dos dados sintéticos e utilizar este tipo de série temporal. Esta ``cultura'' normalmente
permanece invisível na hora de relatar os resultados obtidos em uma pesquisa, mas é de fundamental importância para a realização de trabalhos futuros na área.

Finalmente, como linhas de pesquisa para trabalhos futuros sugere-se o desenvolvimento de um modelo dinâmico mínimo para descrever o comportamente das estruturas coerentes na copa da floresta, sob condições convectivas. Um modelo do tipo instabilidade de Kelvin-Helmholtz, que reconhecidamente apresenta propriedades caóticas~\cite{malik/92}, seria um bom ponto de partida. A análise de mais séries turbulentas que apresentem uma estrutura do tipo rampa, sob diferentes condições experimentais, é também uma continuação natural deste trabalho.