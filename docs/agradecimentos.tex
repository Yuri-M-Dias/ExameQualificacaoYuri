%%%%%%%%%%%%%%%%%%%%%%%%%%%%%%%%%%%%%%%%%%%%%%%%%%%%%%%%%%%%%%%%%%%%%%%%%%%%%%%%
% AGRADECIMENTOS %% opcional

\begin{agradecimentos}  %% insira abaixo seus agradecimentos

\hypertarget{estilo:agradecimentos}{} %% uso para este Guia

Primeiramente gostaria de agradecer a toda a minha família, sem a qual nada disso seria possível, em especial a minha mãe Xarlene e minha madrinha Araída por sempre acreditarem em mim, minha irmã favorita do mundo Natália, meu pai Irair, minhas avós Genoveva e Maria das Graças e meu padrasto Eudes.
Também a todos os inúmeros parentes que me acolheram de alguma forma, seja no Tocantins, em Goiás, no Distrito Federal ou em São Paulo.
Em especial ao meu avô Antônio Macena, que nos deixou muito cedo em um acidente logo após o início deste trabalho.

Ao Dr. Maurício, por me acolher no INPE, aceitado como aluno e me abrir as portas do CCS, meu sincero obrigado pela oportunidade e por todo o suporte que me forneceu.

Ao Dr. Rodrigo, por aceitar um completo desconhecido como aluno, e me ensinar tantas coisas sobre computação ao longo desse tempo, e pelos puxões de orelha merecidos.

Aos meus colegas Bruno e Gabriela que aceitaram dividir apartamento comigo, e me aguentarem por todo esse tempo.

Ao Ítalo, Isomar, Danilo e Johnathan pela amizade, tantos almoços compartilhados e por serem estarem disponíveis para uma conversa aleatória sobre algum conceito espacial obscuro de um manual da União Soviética dos anos 70.

As comissões organizadoras do WETE e do CubeDesign que me permitiram ajudar a organizar esses eventos incríveis, e pelas amizades feitas quando todos trabalham por um mesmo objetivo.

Ao Jun, Pascote, Maria do Carmo, secretarias e todos os trabalhadores do CCS, pelas imensa ajuda dentro do prédio do CCS ao longo dos anos e por sempre proverem o melhor suporte para quem está perdido.

Aos seguranças do INPE, em especial ao Eduardo pelas conversas que passavam da meia noite.

Aos membros do projeto CITAR por compartilharem tantos almoços e piadas, bem como informações importantes da área.
Nunca iria acreditar que questões do StackOverflow poderiam acabar em código de míssil sem vocês.

A todos os membros da biblioteca do INPE, que ao longo dos anos proveram suporte para encontrar os melhores livros, e aguentaram minhas constantes visitas para renovar livros.

Ao INPE e todos os funcionários que proveram todas a infraestrutura necessária para este trabalho, em especial as secretarias da pós-graduação que estão sempre disponíveis para responder perguntas.

\begin{CJK}{UTF8}{min}
  我慢してくれた雑種に心から感謝します.
\end{CJK}

E finalmente, a Coordenação de Aperfeiçoamento de Pessoal de Nível Superior (CAPES) pela bolsa de estudos para executar este trabalho.

\end{agradecimentos}


