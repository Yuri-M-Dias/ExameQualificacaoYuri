%%%%%%%%%%%%%%%%%%%%%%%%%%%%%%%%%%%%%%%%%%%%%%%%%%%%%%%%%%%%%%%%%%%%%%%%%%%%%%%%
% ABSTRACT


\begin{abstract}

%% neste arquivo abstract.tex
%% o texto do resumo e as palavras-chave têm que ser em Inglês para os documentos escritos em Português
%% o texto do resumo e as palavras-chave têm que ser em Português para os documentos escritos em Inglês
%% os nomes dos comandos \begin{abstract}, \end{abstract}, \keywords e \palavrachave não devem ser alterados

%\selectlanguage{english}	%% para os documentos escritos em Português
\selectlanguage{portuguese}	%% para os documentos escritos em Inglês

\hypertarget{estilo:abstract}{} %% uso para este Guia

Satélites são monitorados pelas equipes de solo via pacotes de telemetria, que informam o estado atual dos equipamentos e permitem avaliar a capacidade do satélite de continuar a sua missão.
Esses pacotes de telemetria constituem um corpo de dados de elevado tamanho e complexidade, com satélites que são operados por vários anos geram dados históricos de grande volume, ainda úteis para as atividades de operação e que necessitam de ser arquivados.
O volume de dados históricos de telemetria disponíveis ao Instituto Nacional de Pesquisas Espaciais (INPE) atualmente é estimado em ao menos 3 \textit{terabytes} no total, com tendência a crescer nos próximos anos.
Com este volume, e considerando que as análises de dados sobre esse arquivos não é trivial, necessitando de conhecimento especialista de engenharia, é necessário a implementação de sistemas para realizar consultas e análises sobre esses dados.
Neste trabalho é feita a identificação das consultas que são de interesse dos operadores de satélite, é criada uma modelagem multidimensional para os dados de telemetria utilizando de cubo de dados e então o algoritmo de computação do cubo de dados Frag-Cubing é utilizado como base para implementação.
Primeiramente uma abordagem de pré-processamento das consultas selecionados é implementada, onde as dimensões relacionadas a consulta são filtradas e cubos de baixa dimensionalidade são criados à partir delas.
Essa abordagem é comparada com a abordagem de alta dimensionalidade com todas as dimensões disponíveis, e encontra que, conquanto que as consultas sejam restritas as dimensões filtradas, tem uma vantagem de 15\% no tempo de consulta e nos melhores casos consumindo apenas 10\% de memória utilizada pela abordagem de alta dimensionalidade.
Assim, se as consultas tiverem uma dimensionalidade baixa, existe vantagem em utilizar um cubo preprocessado do zero do que executar uma consulta em uma cubo de dados construído com abordagem de alta dimensionalidade.
Depois uma abordagem baseada na alteração do algoritmo de índice invertido do algoritmo Frag-Cubing é experimentalmente validade, que compõe em utilizar da característica de alta sequencialidade de algumas telemetrias de satélite para substituir as listas de identificadores de tuplas (\textit{TID list}) por listas de intervalos.
Essa abordagem sobre os dados de alta dimensionalidade, testada nas consultas definidas pelos operadores anteriormente, usa em média 20\% da memória que a listas tradicional utiliza, e é até 32x mais rápida para responder consultas em dimensões com alta sequencialidade, porém sendo até 4x mais lenta para responder consultas com dimensões com baixa sequencialidade.

% Keywords
\keywords{%
  \palavrachave{Cubo de Dados}%
  \palavrachave{Índice Invertido}%
  \palavrachave{Satélite}%
  \palavrachave{Telemetria}%
  \palavrachave{Operação de Satélites}%
}

%\selectlanguage{portuguese}	%% para os documentos escritos em Português
\selectlanguage{english}	%% para os documentos escritos em Inglês

\end{abstract}
