%%%%%%%%%%%%%%%%%%%%%%%%%%%%%%%%%%%%%%%%%%%%%%%%%%%%%%%%%%%%%%%%%%%%%%%%%%%%%%%%
% RESUMO %% obrigatório

\begin{resumo}

%% neste arquivo resumo.tex
%% o texto do resumo e as palavras-chave têm que ser em Português para os documentos escritos em Português
%% o texto do resumo e as palavras-chave têm que ser em Inglês para os documentos escritos em Inglês
%% os nomes dos comandos \begin{resumo}, \end{resumo}, \palavraschave e \palavrachave não devem ser alterados

\hypertarget{estilo:resumo}{} %% uso para este Guia

Satélites são monitorados pelas equipes de solo via pacotes de telemetria, que informam o estado atual dos equipamentos e permitem avaliar a capacidade do satélite de continuar a sua missão.
Esses pacotes de telemetria constituem um corpo de dados de tamanho e complexidade significativa, sendo que satélites que funcionam por vários anos geram dados históricos de grande volume, ainda úteis para a operação.
Neste artigo apresentamos uma arquitetura baseada em conceitos de Big Data e Business Intelligence para criar uma representação de dados de telemetria pronta para a análise por operadores e engenheiros de satélite no Instituto Nacional de Pesquisas Espaciais (INPE), bem como apresentamos o fluxo de dados utilizado pelos dados históricos de telemetria de um dos satélites operados pelo INPE.

\palavraschave{%
	\palavrachave{Cubo de Dados}%
	\palavrachave{Big Data}%
	\palavrachave{Operação}%
	\palavrachave{Satélite}%
	\palavrachave{Data Warehouse}%
}
 
\end{resumo}
