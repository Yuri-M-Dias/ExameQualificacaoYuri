%%%%%%%%%%%%%%%%%%%%%%%%%%%%%%%%%%%%%%%%%%%%%%%%%%%%%%%%%%%%%%%%%%%%%%%%%%%%%%%%
% RESUMO %% obrigatório

\begin{resumo}

\hypertarget{estilo:resumo}{} %% uso para este Guia

Satélites são monitorados pelas equipes de solo via pacotes de telemetria, que informam o estado atual dos equipamentos e permitem avaliar a capacidade do satélite de continuar a sua missão.
Esses pacotes de telemetria constituem um corpo de dados de tamanho e complexidade significativa, sendo que satélites que funcionam por vários anos geram dados históricos de grande volume, ainda úteis para a operação.
Nesta proposta, apresentamos a estrutura de um Cubo de Dados como uma solução para executar consultas e análises sobre esse volume de dados, que são classificados como \textit{Big Data}.
Apresentamos os conceitos da área de Cubo de Dados relevantes, bem como uma revisão de como outros operadores de satélite estão lidando com volumes de dados nessa classificação.
Devido a característica de alta dimensionalidade dos dados de telemetria, algoritmos clássicos da aŕea do Cubo de Dados teriam dificuldade em ter um resultado satisfatório para os operadores de satélite.
Por isso, neste trabalho, propomos formalizar as consultas que são de interesse dos operadores de satélite, criar uma modelagem multidimensional para os dados de telemetria utilizando de cubo de dados, e avaliar quais são os algoritmos de construção do cubo que conseguiriam suprir as necessidades dos dados.
Também apresentamos os resultados intermediários alcançados, bem como o planejamento para a continuação do trabalho.

\palavraschave{%
	\palavrachave{Cubo de Dados}%
	\palavrachave{Big Data}%
	\palavrachave{Operação}%
	\palavrachave{Satélite}%
	\palavrachave{Data Warehouse}%
}

\end{resumo}
