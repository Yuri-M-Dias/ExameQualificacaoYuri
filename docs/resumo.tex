%%%%%%%%%%%%%%%%%%%%%%%%%%%%%%%%%%%%%%%%%%%%%%%%%%%%%%%%%%%%%%%%%%%%%%%%%%%%%%%%
% RESUMO %% obrigatório

\begin{resumo}

\hypertarget{estilo:resumo}{} %% uso para este Guia

In order to successfully operate a spacecraft, satellite operators need expert knowledge of the spacecraft and how the subsystems are related to and interacts with each other. With data spread over years of operations, it is important to know what questions to ask, and performing analysis on those datasets is not straightforward. In this paper, we present the results of a query selection process with an experienced satellite engineer, applying the most relevant queries to a data cube, measuring the response times and memory consumption. This lead to testing two distinct data cube inputs: the ``high-dimensional'' of building the data cube with all telemetries at input time; and the ``low-dimensional'' of building a subcube with the query telemetries at query time. The tests were executed with the Frag-Cubing data cube algorithm, using historic data from one of the National Institute for Space Research (INPE) satellites, with high-dimensional (over 130 dimensions) and big volume features (over \(\ensuremath{2\times 10^{7}}\) tuples), only with sequential computing of the data cube. The results indicate that the low-dimensional approach reduces the memory consumption to answer the queries in between 33\% and 0,01\% of the high-dimensional approach, while keeping a similar, or in some cases up to 20\% faster, query response time.

\palavraschave{%
  \palavrachave{Data Cube}%
  \palavrachave{Inverted Index}%
  \palavrachave{Satellite}%
  \palavrachave{Telemetry}%
  \palavrachave{Satellite Operations}%
}

\end{resumo}
