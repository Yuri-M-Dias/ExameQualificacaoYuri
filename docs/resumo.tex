%%%%%%%%%%%%%%%%%%%%%%%%%%%%%%%%%%%%%%%%%%%%%%%%%%%%%%%%%%%%%%%%%%%%%%%%%%%%%%%%
% RESUMO %% obrigatório

\begin{resumo}

\hypertarget{estilo:resumo}{} %% uso para este Guia

Satélites são monitorados pelas equipes de solo via pacotes de telemetria, que informam o estado atual dos equipamentos e permitem avaliar a capacidade do satélite de continuar a sua missão.
Esses pacotes de telemetria constituem um corpo de dados de tamanho e alta complexidade, sendo que satélites que operados por vários anos geram dados históricos de grande volume, ainda úteis para as atividades de operação.
O volume de dados históricos de telemetria disponíveis ao INPE atualmente é estimado em ao menos 3 \textit{terabytes} no total, com tendência a crescer nos próximos anos.
Esta proposta apresenta o uso de Cubo de Dados como solução para executar consultas e análises sobre esses dados.
Os conceitos da área de Cubo de Dados são apresentados, bem como uma revisão de como outros operadores de satélite estão lidando com grandes volumes, variedades e velocidade de atualização de dados, cenário que define um contexto de \textit{Big Data} para o domínio de controle de satélites.
Devido a característica de alta dimensionalidade dos dados de telemetria, algoritmos clássicos da aŕea do Cubo de Dados tem dificuldade em responder consultas com resultado satisfatório para os operadores de satélite.
Assim, neste trabalho é proposto identificar as consultas que são de interesse dos operadores de satélite, criar uma modelagem multidimensional para os dados de telemetria utilizando de cubo de dados, e avaliar quais são os algoritmos de construção do cubo que conseguiriam suprir as necessidades dos dados.
Também são apresentados os resultados alcançados até o momento, bem como o planejamento para a continuação do trabalho.

\palavraschave{%
	\palavrachave{Cubo de Dados}%
	\palavrachave{\textit{Big Data}}%
	\palavrachave{Satélites}%
	\palavrachave{Telemetrias}%
	\palavrachave{Operação de Satélites}%
}

\end{resumo}
