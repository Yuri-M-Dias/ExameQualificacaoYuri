%%%%%%%%%%%%%%%%%%%%%%%%%%%%%%%%%%%%%%%%%%%%%%%%%%%%%%%%%%%%%%%%%%%%%%%%%%%%%%%

\chapter{CONCLUSÃO E TRABALHOS FUTUROS}

Este trabalho apresenta uma abordagem de cubo de dados para executar operações de análise nos dados de telemetrias de satélites. Essa abordagem utiliza de conceitos de \textit{Big Data} para orientar a execução de consultas em dados com muitas dimensões e de alta complexidade. Uma revisão da literatura de arquiteturas de \textit{Big Data} é apresentada, demonstrando que tipos de tecnologias e abordagens estão em uso por outros operadores de satélite.

Também são apresentados resultados intermediários de análises e softwares feitos para a análise primária dos dados de telemetria. 

\textbf{Como essa arquitetura é melhor(diferente?) da utilizada por outros operadores? O que o Cubo de Dados traz de diferente?}

\textbf{Planos futuros, o que vai ser implementado daqui para frente}

\textbf{Queries interessantes dos operadores?}

\textbf{Implementação do cubo de dados?}

