%%%%%%%%%%%%%%%%%%%%%%%%%%%%%%%%%%%%%%%%%%%%%%%%%%%%%%%%%%%%%%%%%%%%%%%%%%%%%%%

\chapter{CONCLUSÃO}\label{ch:concl}

Este trabalho apresenta uma abordagem de cubo de dados para executar operações de análise nos dados de telemetrias de satélites.
Essa abordagem utiliza de conceitos de \textit{Big Data} para orientar a execução de consultas em dados com muitas dimensões, alto número de tuplas e alto \textit{skew}.
{\color{cerulean}
Uma revisão da literatura de arquiteturas de \textit{Big Data} é apresentada, demonstrando que tipos de tecnologias e abordagens estão em uso por outros operadores de satélite, bem como uma revisão sobre os conceitos e abordagens de cubo de dados.
Uma estrutura de utilização do cubo de dados é apresentada, se baseando em uma extensão ao algoritmo \textit{FragCubing} de computação do cubo de dados.
}

São apresentados resultados intermediários de análises dos dados de telemetria, bem como os produtos de software utilizados para realizar as análises e algumas descobertas desse processo.
Esses resultados mostram que a aplicação do cubo é relevante para os dados disponíveis, bem como que é possível implementar algumas etapas do fluxo de dados, propriamente adequadas para lidar com \textit{Big Data}.

\section{Planejamento}\label{ch:concl:planning}

Para o trabalho da dissertação, os passos seguintes são:

\begin{enumerate}
\item Formalizar quais são as consultas relevantes para os operadores de satélite, e quais são as atividades de análise que podem ser expressas como consultas;
\item Criar uma representação dimensional do cubo de dados apropriada para as consultas identificadas, mapeando as medidas que são necessárias e quais os seus tipos;
\item Implementar a representação com as medidas em vários algoritmos da literatura recente, mais notadamente os revisados no capítulo~\ref{ch:corr:cube} e coletar os resultados da execução das consultas relevantes para os operadores;
\item Avaliar os resultados da implementação dos algoritmos e mostrar qual das abordagens é mais apropriada para o cenário da operação.
\end{enumerate}

O passo \textit{c} está parcialmente implementado no pacote mostrado na seção~\ref{ch:impl:rfrag}, porém precisa de trabalho significativo de implementação para executar outros algoritmos e realizar os testes das consultas relevantes.

Os resultados esperados da dissertação seriam: o mapeamento das consultas para os operadores de satélite e a sua representação em um cubo de dados, que teria resultados de implementações diferentes para conseguir avaliar qual dos algoritmos disponíveis é o mais adequado para o cenário de operação.

%\subsection{Trabalhos futuros}
%\label{ch:concl:planning:future}

%Como trabalhos futuros, a arquitetura do fluxo de dados mostrada na seção~\ref{ch:prop:dataflow} pode ser implementado nos moldes das outras agências a exemplo de~\ref{ch:corr:ops}, numa arquitetura que permita a inclusão de todos os tipos de dados elecandos neste trabalho na seção~\ref{ch:prop:data}.
%Também seria interessante expandir os tipos de algoritmos que serão testados para esse trabalho.

%Expandir o uso dos dados para a abordagem de cubo seria relevante, pois existem desafios diferentes quando se lida com um satélite de tamanho grande (ex. GEO) e uma constelação de satélites menores, porém com um volume de dados comparável.
%Alguma abordagem para lidar com dados de CubeSats seria relevante para o momento, principalmente se forem de diferentes cubesats e/ou de constelações.

