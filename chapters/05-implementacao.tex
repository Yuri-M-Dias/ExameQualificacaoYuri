%%%%%%%%%%%%%%%%%%%%%%%%%%%%%%%%%%%%%%%%%%%%%%%%%%%%%%%%%%%%%%%%%%%%%%%%%%%%%%%

\chapter{IMPLEMENTAÇÃO E RESULTADOS}
\label{ch:impl}

Para a implementação da arquitetura, é necessário conhecimento sobre o domínio dos dados e como eles estão organizados.
Para esse estágio da pesquisa, foram utilizados dados do SCD-2 fornecidos pelo CCS, porém alguns softwares foram implementados como parte da pesquisa sobre o cubo de dados e como resultados da análise dos dados.

\section{SCD-Dashboard}
\label{ch:impl:dash}

Os dados vindos do SCD2 possuem 135 dimensões e estão distribuídos em um período de 4 anos, tornando a sua análise não trivial.
Durante atividades de \textit{Data Science} era necessário criar muitos gráficos e visualizações com períodos e dimensões variadas, e com tantas dimensões isso levava vários pedaços de código copiados para criar relatórios sobre os dados e dimensões.

Para facilitar essa análise e resolver esse problema, foi criado um software chamada de \textbf{SCD-dashboard} para facilitar na visualização dos dados de telemetria.
Ele foi implementado utilizando o pacote Shiny~\cite{} da linguagem R~\cite{rcoreteamLanguageEnvironmentStatistical2018}, que já estava sendo utilizada para criar as visualizações, a interface permite apenas criar as visualizações e aplicar os outros algoritmos relacionados de uma interface amigável e sem ter a necessidade de escrever código.

Ele funciona sobre um banco de dados PostgreSQL que possui o histórico das telemetrias já importadas e propriamente transformadas por outro script feito só para os dados do SCD2, porém devem funcionar para qualquer dado exportado via CSV pelo SatCS.

Essa ferramenta também pode ser vista como um piloto para implementar algo semelhante as \textit{dashboards} criadas por outras agências, sendo que tem um precedente na ferramenta MARTE utilizada pela NASA~\cite{fernandezTelemetryAnomalyDetection2017}, que utiliza das mesmas tecnologias e conceitos, porém focada no algoritmo de detecção de anomalias.
Ferramentas mais parecidas, e mais completas, estão no CHART e nas dashboard criadas usando o Kibana em~\cite{mateikUsingBigData2017} e~\cite{zhangBigDataFramework2017}.

\begin{figure}[ht]
	\caption{\color{red} Aparência do SCD-Dashboard [TODO]}
	\vspace{6mm}
	\begin{center}
		\resizebox{10.5cm}{!}{\includegraphics{Figuras/mapalbacorr.png}}
	\end{center}
	\vspace{4mm}
	\legenda{}
	\label{fig:scddashboard}
	\FONTE{Autor (2019)}
\end{figure}

Essa ferramenta foi utilizada extensamente para visualização dos dados e para análise exploratória, sendo que foi melhorada durante as disciplinas de \textit{Data Science} e Algoritmos de \textit{Data Mining}, com apresentações da mesma feitas ao vivo.

\section{RFragCubing}
\label{ch:impl:rfrag}

O algoritmo FragCubing, apresentado na seção~\ref{ch:corr:cube:frag}, foi disponibilizado em forma compilada via código de C++, porém, ele foi feito com uma interface complicada de ser utilizada e automatizada, e não permitia a importação dos dados de telemetria sem um trabalho de pré-processamento não trivial antes.
Como essa implementação foi a utilizada no trabalho de~\cite{silva:2015:abordagensParaCubo}, os resultados possuem histórico de comparação facilitada, portanto o seu uso contínuo seria interessante.

Para isso, foi criado um pacote na linguagem R chamado de \textbf{RFragCubing}, que permite a integração com o algoritmo implementado em C++.
Esse pacote faz a interface com o código do FragCubing, permitindo importar os dados, executar as queries e retornar os resultados das mesmas, bem como algumas adições de medidação de memória e tempo de processamento.
Uma das vantagens do pacote está na distribuição facilitada, com uma interface que permite a execução da mesma consulta em outros algoritmos de construção do cubo, uma das razões para fazer a implementação, para facilitar a comparação entre os algoritmos.

\begin{figure}[ht]
	\caption{\color{red} Aparência da interface Shiny do RFragCubing [TODO]}
	\vspace{6mm}
	\begin{center}
		\resizebox{10.5cm}{!}{\includegraphics{Figuras/mapalbacorr.png}}
	\end{center}
	\vspace{4mm}
	\legenda{}
	\label{fig:shinyrfrag}
	\FONTE{Autor (2019)}
\end{figure}

\section{Medida de Similaridade}
\label{ch:impl:similarity}

Utilizando a ferramenta criada em~\ref{ch:impl:dash} para uma atividade de exploração de dados que envolvia realizar consultas multidimensionais baseadas no conceito do cubo de dados, um padrão foi notado entre as telemetrias: quando uma medida de agregação por contagem era executada sobre telemetrias que se sabia ter algum tipo de relacionamento entre elas, elas possuiam uma curva característica, como da figura~\ref{fig:scdsimilaritygraph}.

\begin{figure}[ht]
	\caption{\color{red} Curva de de agregação gerada pela SCD-Dashboard [TODO]}
	\vspace{6mm}
	\begin{center}
		\resizebox{10.5cm}{!}{\includegraphics{Figuras/mapalbacorr.png}}
	\end{center}
	\vspace{4mm}
	\legenda{}
	\label{fig:scdsimilaritygraph}
	\FONTE{Autor (2019)}
\end{figure}

Essa curva foi utilizada para desenvolver um algoritmo de classificação do relacionamento entre as telemetrias, pois possuia algumas vantagens: a medida de contagem dos valores que se repetem é independente do tipo dos valores em si, permitindo comparar um valor contínuo com um discreto bem como discretos com discretos e contínuos com contínuos, de equipamentos diferentes e com características diferentes; é uma medida que funciona com qualquer número de dimensões, sendo uma operação $\Theta(D)$, com $D$ sendo o número de dimensões; 

\begin{figure}[ht]
	\caption{\color{red} Interface de Medição de Relacionamento [TODO]}
	\vspace{6mm}
	\begin{center}
		\resizebox{10.5cm}{!}{\includegraphics{Figuras/mapalbacorr.png}}
	\end{center}
	\vspace{4mm}
	\legenda{}
	\label{fig:similarityshiny}
	\FONTE{Autor (2019)}
\end{figure}

