%%%%%%%%%%%%%%%%%%%%%%%%%%%%%%%%%%%%%%%%%%%%%%%%%%%%%%%%%%%%%%%%%%%%%%%%%%%%%%%

\chapter{PROPOSTA}\label{ch:prop}

{\color{red}
Nesta seção apresentamos os dados que serão utilizados e são relevantes para a operação, bem como apresentamos o fluxo de dados utilizado nos trabalhos correlatos e a arquitetura proposta, por fim explicando o cubo de dados que será implementado.
}

\section{Objetivos}\label{ch:prop:obj}

{\color{red} Research questions?!}

\section{Dados no INPE}\label{ch:prop:data}

{\color{red} Estimativa dos dados, crescimento deles no INPE utilizando do disponível pro SCD e estimativas do CBERS+Amazônia}

\section{Cubo de Dados}\label{ch:prop:cubearch}

A figura~\ref{fig:cubearch} demonstra a divisão em 4 camadas de uma estrutura de Cubo de Dados.
Essas camadas demonstram tudo o que é necessário para a implementação de um Cubo de Dados, não sendo necessário que uma camada esteja fortemente atrelada a outra.

\begin{figure}[ht]
	\caption{Estrutura do Cubo de dados}\label{fig:cubearch}
	\vspace{6mm}
	\begin{center}
		\resizebox{10cm}{!}{\includegraphics{Figuras/DataCubeArchitecture.pdf}}
	\end{center}
	\vspace{2mm}
	\legenda{}
	\FONTE{Produção dos autores}.
\end{figure}

Para esta proposta, vamos nos concentrar apenas na proposição de um algoritmo de computação do cubo de dados mais apropriado, utilizando das outras seções quando elas vão se tornando necessárias.
Os detalhes, algoritmos e conceitos listados na figura estão majoritariamente descritos na seção~\ref{ch:fun:cube}.

Uma informação interessante é que esta estrutura mostra o uso de pelo menos duas linguagens de computação sobre os dados: uma é a Cube Query Language que será utilizada pelo usuário para realizar as operações sobre o cubo (Drill-Down, Roll-up, etc), e outra é a linguagem que será utilizada pelo cubo para realizar essas operações, e elas podem ser independentes, por exemplo, pode-se utilizar SQL extendida com vocabulário de OLAP, porém o algoritmo de cubo de dados internamente pode consultar uma estrutura feita com MapReduce para o cálculo das medidas e das agregações.

Porém, utilizar duas linguagens muito diferentes nesse ponto pode não ser uma boa ideia, pois adicionaria um nível de diferença entre o usuário e os dados.
Caso seja necessário realizar uma consulta OLTP normal, sem o uso do cubo de dados, por exemplo, essa diferença ficaria mais óbvia, por exemplo traduzir uma consulta de SQL para MapReduce não seria muito fácil simplesmente por ter que entender de ambas as linguagens bem para conseguir fazer isso.
Deste modo, é interessante manter a mesma linguagem ao longo da estrutura, apenas alterando nas operações relevantes para o cubo de dados.

Com isso se torna necessário ressaltar o último nível, a Base de Dados: a escolha de banco de dados vai impactar como o algoritmo funciona, visto que existem diferentes sistemas de arquivos e como eles são atingidos, bem como o estilo do banco vai mudar como o algoritmo deve gerar o cubo, pois a base pode utilizar diferentes paradigmas de banco de dados~\cite{cuzzocreaDataWarehousingOLAP2013}.

\subsection{Algoritmos de construção do cubo}\label{ch:prop:cubearch:algo}

Uma das necessidades de usar algoritmos diferentes de cubo de dados está no número de dimensões que um certo cubo consegue realizar pesquisas: consultas com mais que 15 dimensões não são comummente(ou praticamente) executadas em alguns algoritmos, como o trabalho de~\cite{silva:2015:abordagensParaCubo} demonstra.

{\color{red}
Como estabelecido na seção~\ref{ch:prop:data}, os dados de telemetria de interesse possuem muito mais do que o limite de consultas em até 60 dimensões: com mais de 130 telemetrias para os satélites da família da SCD, e milhares para satélites maiores, a execução de consultas complexas seria normalmente inviável nos algortimos de construção do cubo.
}
Esse problema é geralmente resolvido pela modelagem dimensional, como em~\cite{AzevedoAmbr:2010:ArSaTe}, porém isso transforma os dados de um formato ``largo'' para um formato ``longo'', aumentando o número de tuplas.

Deste modo é necessário investigar as abordagens de construção de cubo que funcionem com muitas dimensões, e que permitam o cálculo das medidas necessárias para a operação.
Neste trabalho, iremos investigar algumas dessas abordagens e como elas se comparam para a análise.

{\color{red} Desenvolver mais sobre o algoritmos?}

