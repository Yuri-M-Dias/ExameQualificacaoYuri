%%%%%%%%%%%%%%%%%%%%%%%%%%%%%%%%%%%%%%%%%%%%%%%%%%%%%%%%%%%%%%%%%%%%%%%%%%%%%%%

\chapter{Analysis and Discussion}\label{ch:analysis}

In this chapter, a critical analysis of the algorithms is presented, as well as an overview of how useful are the results and what are their shortcomings.
The results from chapters~\ref{ch:interval} and~\ref{ch:querypart} show that simple approaches can be used to enhance the query response time for the selected queries, and can be easily ported to other domains and styles of computation.

\autoref{tab:analysis_overview} shows the characteristics in which each algorithm has showed to excel at.
Frag-Cubing is still preferred when the data has a low degree of sequentiality, as there's little advantage in using the IntervalFrag scheme when the intervals are closer to the size of the original list.
On those cases, IntervalFrag is discouraged, as the algorithm will be slower than Frag-Cubing's by simple virtue of needing more instructions to answer the same query, being up to 400\% slower than the same query under Frag-Cubing.

\begin{table}[!ht]
  \centering
  \caption{Preferred algorithm to use}\label{tab:analysis_overview}
  \footnotesize
  \begin{tabular}{|C{1.8cm}|C{2.3cm}|C{2.3cm}|C{2.65cm}|C{2cm}|C{1.6cm}|}
    \hline
    & \bfseries Low Sequentiality &\bfseries High Sequentiality &\bfseries High Dimensionality &\bfseries High Cardinality &\bfseries High Skew \\
    \hline
    Computing the base cube & IntervalFrag & IntervalFrag & IntervalFrag & IntervalFrag& IntervalFrag \\
    \hline
    Subcube query & Frag-Cubing & IntervalFrag & IntervalFrag & Frag-Cubing$^1$ & FragCubing \\
    \hline
    Point query & Frag-Cubing & IntervalFrag & IntervalFrag & Frag-Cubing$^1$ & FragCubing \\
    \hline
    Low available RAM & IntervalFrag & IntervalFrag & IntervalFrag & IntervalFrag & IntervalFrag \\
    \hline
    Fast Query Reponse & Frag-Cubing & IntervalFrag & Frag-Cubing$^1$ & Frag-Cubing$^1$ & Frag-Cubing \\
    \hline
    \multicolumn{6}{l}{$^{1}$\footnotesize{If there's high sequentiality, prefer IntervalFrag}} \\
  \end{tabular}
\end{table}
\normalsize

In summary, in case the RAM available is low and the worst case query response times can be up to 4x slower than Frag-Cubing's, then IntervalFrag should be preferred as the main method of computing the data cube.
In case the data have a very low sequentiality, and RAM is available, then using Frag-Cubing should still be preferred for the faster response times.
IntervalFrag will excel at any dimension that has a high sequentiality, even if it also has a high cardinality, however dimensions that have a high cardinality will tend to have a low sequentiality in practice, and this usage might be rarer.
The Skew parameter can influence the algorithm both ways, as it does not necessarily mean that the dimension will have a higher or lower sequentiality and cardinalities.

When the dimensions have a high degree of sequentiality, then IntervalFrag excels, as it can not only answer the same queries much faster, but also using only a fraction of the memory used by Frag-Cubing.
Furthermore, Frag-Cubing used much less memory to answer queries $Q1$, $Q2$ and $Q5$, with $Q4$ having a small difference and $Q3$ having no difference in memory usage in the end, when compared with cubes $C1$ to $C5$.
All queries executed on the $C0$ cube with all dimensions used only a fraction of the memory needed to answer queries with IntervalFrag, they were however in general slower to answer.

From the tests made using Frag-Cubing and the different cubes ($C1$ to $C5$) tailored to specific queries, it was shown that the best algorithms can be further enhanced by doing some simple pre-processing of the queries, and depending on the type of query used they can drastically improve upon memory usage requirements, allowing for some frequent queries to be optimized and even allowing for queries that could not be answered under a $C0$ cube to be answered by smaller cubes.
In chapter~\ref{ch:querypart} it was shown that it is faster to load a smaller subset of the data in memory as prepared files when needed and then computing the answer from that file instead of querying a cube that was already loaded in memory, but that used the full dimensional capability of the data.

\RED{ADD AT LEAST ONE GRAPH OR TABLE COMPARING THEM. THE ONE USED FOR THE PRESENTATION IS FINE}

It is important to note IntervalFrag had faster file reading speeds (about 12\% faster) due to improvements made on the implementation, as well as a slightly more efficient set intersection algorithm, which were not backported to Frag-Cubing.
This was done to preserve the Frag-Cubing algorithm's performance, as the original code was made for the C language in 2002 and the updated IntervalFrag implementation uses C++17 standards.
Nonetheless, it was possible to compile Frag-Cubing using the same flags as IntervalFrag under the GNU C++ compiler, with minimal performance differences.

The difference in query response times from IntervalFrag and Frag-Cubing, even when using the same intersection algorithm, was due to IntervalFrag having to do more comparisons to answer the same query, and this implementation could not be further optimized without heavily skewing the response times to IntervalFrag's side, by using other techniques that could also be ported to Frag-Cubing trivially.
Further details on the intersection algorithms tested and their performance differences can be found on Appendix~\ref{ap:a:problem}.

