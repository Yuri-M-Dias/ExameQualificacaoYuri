%%%%%%%%%%%%%%%%%%%%%%%%%%%%%%%%%%%%%%%%%%%%%%%%%%%%%%%%%%%%%%%%%%%%%%%%%%%%%%%

\chapter{Analysis and Discussion}\label{ch:analysis}

In this chapter, a critical analysis of the algorithms is presented, as well as an overview of how useful are the results and what are their shortcomings.
The results from chapters~\ref{ch:interval} and~\ref{ch:querypart} show that simple approaches can be used to enhance the query response time for the selected queries, and can be easily ported to other domains and styles of computation.

\autoref{tab:analysis_overview} shows the characteristics in which each algorithm has showed to excel at.
FragCubing is still preferred when the data has a low degree of sequentiality, as there's little advantage in using the IntervalFrag scheme when the intervals are closer to the size of the original list.
On those cases, IntervalFrag is discouraged, as the algorithm will be slower than FragCubing's by simple virtue of needing more instructions to answer the same query, being up to 400\% slower than the same query under FragCubing.

When the dimensions have a high degree of sequentiality, then IntervalFrag excels, as it can not only answer the same queries much faster, but also using only a fraction of the memory used by FragCubing.
Furthermore, FragCubing used much less memory to answer queries $Q1$, $Q2$ and $Q5$, with $Q4$ having less adavantages and $Q3$ having no difference in memory usage in the end.

\begin{table}[!ht]
  \begin{center}
    \caption{Set Intersection Results, in milliseconds}\label{tab:setresults}
    \begin{tabular}{|c|c|c|c|c|c|c|}
      \hline
      \textbf{Algorithm - N} & \bfseries $2\times10^6$ & \bfseries $4\times10^6$ & \bfseries $6\times10^6$ & \bfseries $8\times10^6$ & \bfseries $1\times10^7$\\
      \hline
      UnorderedSet & 867,802 & 1806,19 & 2586,04 & 3448,11 & 4213,15\\
      \hline
      Scalar & 26,57 & 51,346 & 71,109 & 85,531 & 118,114\\
      \hline
      Li & 26,531 & 45,596 & 66,19 & 86,234 & 108,603\\
      \hline
      BinaryLi & 21,601 & 39,776 & 59,27 & 79,416 & 97,155\\
      \hline
      std::set\_intersect & 17,125 & 34,392 & 51,682 & 67,717 & 84,933\\
      \hline
      SIMD (SS2) & 10,941 & 21,854 & 33,866 & 43,739 & 55,814\\
      \hline
    \end{tabular}
  \end{center}
\end{table}

