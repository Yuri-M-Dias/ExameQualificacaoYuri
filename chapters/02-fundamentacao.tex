%%%%%%%%%%%%%%%%%%%%%%%%%%%%%%%%%%%%%%%%%%%%%%%%%%%%%%%%%%%%%%%%%%%%%%%%%%%%%%%

\chapter{FUNDAMENTAÇÃO}

Neste capítulo, vamos apresentar os conceitos relevantes para este documento, bem como apresentação uma revisão da literatura na área de \textit{Big Data}, terminando com uma visão geral de como o conceito e as tecnologias recentes estão sendo utilizadas por variadas agências espaciais.

\section{Conceitos}

\section{Big Data}

\section{Operação}

\section{Outras Agências}

A tabela \ref{table:bigdataoperators} mostra uma revisão feita em artigos recentes sobre os operadores de satélite e quais tecnologias eles estão utilizando, como demonstrado pelos artigos publicados.

\begin{table}[!ht]%[htbp]
  \begin{center}
  \caption{Operadores e Arquiteturas de Big Data}
  \begin{tabular}{|m{8em}|m{6em}|c|m{10em}|}
			\hline
			Referência & Operador & Ferramenta & Tecnologias \\
			\hline
			\cite{adamskiDataAnalyticsLarge2016} & L3 (EUA) & InControl & Hadoop, Spark, HBase, MongoDB, Cassandra, Amazon AWS \\
			\hline
			\cite{boussoufBigDataBased2018} & Airbus & Dynaworks & Hadoop, Spark, HDFS, HBase, PARQUET, HIVE \\
			\hline
			\cite{schulsterCHARTingFutureOffline2018} & EUMETSAT & CHART & MATLAB, MySQL, Oracle \\
			\hline
			\cite{zhangBigDataFramework2017} & SISET (China) & - & Hadoop, HDFS, PostgreSQL, MongoDB, Logstash, Kibana, ElasticSearch, Kafka, MapReduce \\
			\hline
			\cite{yvernesCopernicusGroundSegment2018} & Telespazio France & PDGS & OLAP (DataCube), Saiku, Pentaho, Jaspersoft OLAP \\
			\hline
			\cite{dischnerCYGNSSMOCMeeting2016} & SwRI + NOAA & CYGNSS MOC & -, SFTP \\
			\hline
			\cite{edwardsDealingBigData2018} & EUMETSAT & MASIF & FTP, RESTful service, JMS Messague Quee, PostgreSQL, vários proprietários* \\
			\hline
			\cite{evansDataMiningDrastically2016} & S.A.T.E + ESA/ESOC & - & Java, CSV, algoritmos proprietários* \\
			\hline
			\cite{fenManagementOperationCommunication2016} & CSMT\& (China) & - & não menciona as tecnologias, mas menciona os problemas em comum \\
			\hline
			\cite{trollopeAnalysisAutomatedTechniques2018} & EUMETSAT & CHART & algoritmos, estudo de caso \\
			\hline
			\cite{gillesFlyingLargeConstellations2016} & L-3 & InControl & Amazon EC2, LXC, Nagios, repetição do primeiro \\
			\hline
			\cite{highsmithSpaceLaunchSystem2015} & Boeing + NASA & - & lançadores, não é o foco da arquitetura \\
			\hline
			\cite{hennionBigdataSatelliteYearly2018} & Thales Alenia & AGYR & Logstash, Kafka, InfluxDB, ElasticSearch, Kibana, Grafana \\
			\hline
			\cite{mateikUsingBigData2017} & Stinger, NASA & - & Logstash, ElasticSearch, Kibana, HDF5, CSV, R, Python, AWS, Excel \\
			\hline
			\cite{fernandezTelemetryAnomalyDetection2017} & NASA & MARTE & R, CSV, ad-hoc \\
			\hline
    \end{tabular}
    \end{center}
 %\FONTE{Coloque a fonte de referência aqui, se houver.}
	\label{table:bigdataoperators}
\end{table}


