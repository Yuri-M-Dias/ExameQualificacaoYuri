%%%%%%%%%%%%%%%%%%%%%%%%%%%%%%%%%%%%%%%%%%%%%%%%%%%%%%%%%%%%%%%%%%%%%%%%%%%%%%%

\chapter{INTRODUÇÃO}

\textbf{[Resto da introdução]}

Os capítulos restantes desta dissertação estão organizados da seguinte maneira:
\begin{itemize}
	\item{Capítulo 2}: Este capítulo apresenta os conceitos e fundamentos teóricos, como apresentando os conceitos do Cubo de Dados, as definições utilizadas de \textit{Big Data}, a definição do problema para a operação.
	\item{Capítulo 3}: Neste capítulo os trabalhos correlatos de Cubo de Dados são apresentados, bem como outros operadores de satélite estão resolvendo os problemas identificados.
	\item{Capítulo 4}: Neste capítulo a arquitetura proposta é apresentada e seus conceitos principais explicados, bem como o fluxo de dados atual do CCS e como a nova arquitetura vai melhorá-lo.
	\item{Capítulo 5}: Esse capítulo apresenta alguns resultados já alcançados, demonstrando os softwares que já foram escritos e as análises que já foram executadas.
	\item{Capítulo 6}: Com base na arquitetura proposta e nos resultados intermediários alcançados, esse capítulo apresentará as conclusões obtidas bem como as direções de implementação para o resto do trabalho.
\end{itemize}

