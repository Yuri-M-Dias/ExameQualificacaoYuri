%%%%%%%%%%%%%%%%%%%%%%%%%%%%%%%%%%%%%%%%%%%%%%%%%%%%%%%%%%%%%%%%%%%%%%%%%%%%%%%

\chapter{INTRODUÇÃO}

Os capítulos restantes desta dissertação estão organizados da seguinte maneira:
\begin{itemize}
\item{Capítulo 2}: Este capítulo apresenta os conceitos e fundamentos correlatos, como apresentando os conceitos do Cubo de Dados, as definições utilizadas de \textit{Big Data}, a definição do problema para a operação e como as outras agências espaciais e operadores estão utilizando essses conceitos com base na literatura recente.
\item{Capítulo 3}: Neste capítulo a arquitetura proposta é apresentada e seus conceitos principais explicados, bem como o fluxo de dados atual do CCS e como a nova arquitetura vai melhorá-lo.
\item{Capítulo 4}: Esse capítulo apresenta alguns resultados já alcançados, demonstrando os softwares que já foram escritos e as análises que já foram executadas.
\item{Capítulo 5}: Com base na arquitetura proposta e nos resultados intermediários alcançados, esse capítulo apresentará as conclusões obtidas bem como as direções de implementação para o resto do trabalho de mestrado.
\end{itemize}

