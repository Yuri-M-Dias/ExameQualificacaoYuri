%%%%%%%%%%%%%%%%%%%%%%%%%%%%%%%%%%%%%%%%%%%%%%%%%%%%%%%%%%%%%%%%%%%%%%%%%%%%%%%

\chapter{INTRODUÇÃO}\label{ch:intro}

O Centro de Controle de Satélites (CCS) localizado no Instituto Nacional de Pesquisas Espaciais (INPE) atualmente monitora e controla alguns satélites: a família do Satélite de Coleta de Dados (SCD), composta de dois satélites SCD-1 e SCD-2, e o Satélite Sino-Brasileiro de Recursos Terrestres (CBERS), com o quinto satélite em operação atualmente, o CBERS-4.
Estes satélites realizam passagens sobre as estações terrenas do INPE, durante o qual o CCS recebe dados do estado do satélite, chamados de telemetrias, e envia telecomando, utilizados para controlar o satélite, bem como realiza atividades de manutenção e estimativa, como medidas de velocidade e posição de cada satélite~\cite{AzevedoAmbr:2010:ArSaTe}.

Dados de telemetria geralmente carregam medidas de sensores e verificações de saúde dos instrumentos, como temperatura das baterias, corrente de algum subsistema, se um dado equipamento está ativo ou não, bem como dados que os operadores e engenheiros acham necessários para a operação, entre outros~\cite{larsonSpaceMissionAnalysis1999}.
Estes dados precisam ser guardados por toda a vida do satélite, sendo que para satélites que estão funcionando por vários anos, eles adquirem um volume considerável, que não pode ser descartado.
No caso dos satélites da família SCD, o SCD-1 já estando operacional por mais de 25 anos, e continuando a gerar dados, com um volume aproximado de 7GB por ano.

Para satélites mais complexos como os da família CBERS, que possuem mais de 4 mil telemetrias sendo rastreadas, geram um volume de dados cuja análise não é trivial, e só pode ser feita por especialistas no funcionamento do satélite.
Com os lançamentos futuros do CBERS-4A e do Amazônia-1, o volume de dados e a complexidade da análise dos mesmos deve aumentar, criando novas necessidades de operação~\cite{JulioFoAmbrFerrLour:2017:ChImSp}.

Considerando as necessidades do CCS, esses dados entram na definição de \textit{Big Data}: possuem um grande volume, são gerados continuamente, possuem formatos diversos, sua análise é de alto valor e existe uma incerteza quanto a qualidade dos dados devido a problemas de comunicação e degradação dos instrumentos.
Esses características são denotadas pelos cinco Vs do \textit{Big Data}: Volume, Variedade, Velocidade, Valor e Veracidade~\cite{kacfahemaniUnderstandableBigData2015}.

Deste modo, é necessário criar uma estrutura que permita a análise e consulta desses dados de uma forma estrutura e que tenha desempenho satisfatório.
As tecnologias de \textit{Data Warehouse} (DW) e \textit{Online Analytical Processing} (OLAP) tem demonstrado capacidade e experiência para atingir esses objetivos~\cite{bimonteOpenIssuesBig2016}, inclusive na área espacial~\cite{yvernesCopernicusGroundSegment2018}.
Essas tecnologias executam a generalização de dados agregando enormes quantidade de dados em vários níveis de abstração, assim tornam elementos essenciais de apoio à decisão e atraem a atenção tanto da indústria como das comunidades de pesquisa.
Sistemas OLAP, que são tipicamente dominados por consultas complexas que envolvem operadores \textit{group-by} e operadores de agregações, são as principais características entre essas ferramentas.

Sistemas OLAP são baseados em um modelo multidimensional chamado de cubo de dados, que é uma generalização do operador \textit{group-by} sobre todas as combinações possíveis das dimensões, com variados níveis de granularidade~\cite{grayDataCubeRelational1996}.
Cada combinação é chamada de um subcubo, que correspondem a um conjunto de células descritas como tuplas sobre as dimensões do subcubo.
Além das dimensões, cada tupla contém um fato, também chamado de medida, que representa o que será medido no processo de análise.

Cada dimensão pode estar organizada em uma hierarquia para facilitar a análise.
Por exemplo, uma dimensão tempo pode ser dividida em ``dia < mês < ano'', com ano sendo o nível mais genérico.
Essa prática visa facilitar a interpretação dos dados pelos usuários.
Medidas são atributos atributos associados a uma combinação de dimensões, sendo geradas de forma estatística.

Tecnologias OLAP são caracterizadas pela habilidade em responder consultas de apoio a decisão de forma eficiente.
Para atingir isso, o cubo de dados deve ser materializado antes da execução da consulta.
Isso significa que as combinações de dimensões são computadas previamente, assim gerando o cubo completo.
Porém, essa abordagem possui um custo computacional exponencial em relação ao número de dimensões, assim a materialização completa do cubo envolve um grande número de células e um tempo substancial para a sua execução.

Dados de satélite são caracterizados pela sua alta dimensionalidade, onde um satélite pode precisar rastrear milhares de telemetrias.
Por exemplo, supondo um satélite com $n = 100$ telemetrias, e cada telemetria representando uma dimensão, teremos $2^{100}$ possíveis subcubos para a implementação de um cubo de dados.
Supondo uma cardinalidade, o número de valores diferentes em cada telemetria, como sendo de $100$, teremos $101^{100} \approx 10^{200}$ células para cada dimensão.

Dessa forma, conseguir computar e manter um cubo de dados é um problema exponencial, e reduzir o seu consumo de memória e tempo de computação é de fundamental importância para desenvolver um sistema OLAP.
Para a área espacial essa necessidade é maior: a maior parte dos algoritmos de computação do cubo tem problemas em lidar com mais do que 15 dimensões~\cite{silva:2015:abordagensParaCubo}.
Neste trabalho, visamos aplicar a tecnologia de cubo de dados por meio de algoritmos que consigam responder as consultas da área espacial eficientemente, e consigam lidar com a alta dimensionalidade, velocidade e diversidade dos dados para a operação.

\section{Objetivos}\label{ch:intro:obj}

{\color{red} ?}

\section{Organização da proposta}\label{ch:intro:org}

Os capítulos restantes desta proposta estão organizados da seguinte maneira:

\begin{itemize}
	\item{Capítulo 2}: Este capítulo apresenta os conceitos e fundamentos teóricos desta proposta, como os conceitos relevantes de operação de satélites, \textit{Data Warehouse}, \textit{Big Data} e do Cubo de Dados.
	\item{Capítulo 3}: Neste capítulo os trabalhos correlatos de Cubo de Dados são apresentados, bem como as arquiteturas que outros operadores de satélite estão implementando.
	\item{Capítulo 4}: {\color{red} Neste capítulo a proposta é apresentada e seus conceitos principais explicados, bem como o fluxo de dados atual do CCS e como a nova arquitetura vai melhorá-lo.}
	\item{Capítulo 5}: Esse capítulo apresenta os resultados alcançados até o momento.
	\item{Capítulo 6}: Com base nos resultados intermediários alcançados, esse capítulo apresentará as conclusões obtidas, bem como as direções de implementação para o resto do trabalho.
\end{itemize}

